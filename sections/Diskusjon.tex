\section{Diskusjon}

- Hvor i smarttelefonen er sensoren plassert? \newline 
- Hva kan gjøres forskjellig for å få mer nøyaktige resultater? - Hva kan endres til neste gang? \newline
- Hvilke svakheter ligger i målingene? \newline
- Er det noen målinger som skiller seg ut? Hvorfor? \newline
- Er det nøyaktige resultater i forhold til andre type forsøk? \newline 
- Hva har resultatene å bidra med til framtidige forsøk? \newline                                    
- Bestem aksene (koordinatsystemet) til smarttelefonen. \textbf{se figur \ref{fig:telf_akser} }\newline 
- Hva er oppløsningen til sensoren? \newline 
- Er sensoren kalibrert? \newline
- Er målingene stabile? Ser du variasjoner? Er de påvirket av temperatur, lading
osv...? \textbf{Under den ene målingen når det kjørte en bil ved siden av måleapparatet, viste målingene en liten endring i dataene sammenlignet med de andre målingene. }\newline
- Er resultatene reproduserbare etter en reboot/avstengning? \newline
- Kan omgivelsene påvirke målingene? \newline                      
- Kan du detektere lokale variasjoner? Hva kan forårsake disse variasjonene?  \newline
- Avansert: Kan du oppdage variasjoner under døgnet? \newline
- Avansert: Kan nordlys påvirke målingene? \newline 

\noindent\textbf{Mulige ting som kan ha påvirket resultatene}:
\begin{itemize}
    \item Forskjellige mobiltelefoner som gir forskjellige data 
    \item biler som kjører forbi
    \item andre elektromagnetiske kilder i nærheten? 
    \item trefoten kan ha vært ustødig
    \item telefonholderen
    \item vibrasjoner i bakken               
    \item siktet til vateret, er dette nøyaktig nok?
    \item Ble mobilen lagt nøyaktig nok i forhold til vateret?
    \item Værforhold - det var delvis overskyet, ingen sol
    \item flymodus, restart - hvordan påvirket dette resultatene?

Det er naturlig å sammeligne med verdiene den empiriske modellen World Magnetic Model (WMM) gir.
Magnetisk deklinasjon for Trondheim (hentet 19. april 2023) er +4.63\textdegree. Inklinasjonen er 74.9\textdegree (se figur \ref{fig:WMM}). 
\newline
Det er verdt å merke seg at WMM opererer med positive vinkler \textit{fra} nord \textit{til} retningen på magnetfeltet. 
Teorien, metoden og resultatet presentert i denne rapport bruker positivt fortegn på vinkelen \textit{fra} magnetfeltet \textit{til} geografisk nord.
Ved å definere deklinasjonen som vinkelen \textit{fra} geografisk nord \textit{til} retningen til magnetfeltet i et koordinatsystem med positiv vinkel med klokken sett ovenfra vil fortegnet på resultatene i rapporten stemme overens med fortegnet til WMM.  
\par
Fra figur \ref{fig:plot_declination} og tabell \ref{fig:tabell_deklinasjon} er det tydelig at misvisningen 
\par
Telefonene kan ha et ulikt koordinatsystem enn det Phyphox oppgir. 
Dette vil isåfall være en systematisk feil som påvirker alle målingene. 
????? Dette kan være grunnen til at resultatet ligger  mot en ulik vinkel for hver telefon. 

Det kan også hende at telefonene ikke har ligget helt i vater eller at de ikke har ligget parallelt med telefonholderen. Dette kan da gi feilmargin. For de måleseriene som har samme referansepunkt, er det liten variasjon mellom målingene for samme person. Trolig er det dermed ikke store feil som følge av plasseringen av telefonen i telefonholderen. 

Eksterne forstyrrelser var det i liten grad. I området rundt målingslokasjonen var det ingen store kilder av elektromagnetisk stråling. Dermed er det trolig liten endring i målingene som følge av eksterne kilder. 

Å bruke et vater som siktemiddel gir ikke en stor grad av presisjon og inneholder et menneskelig element. Dermed er det trolig at en ville fått mer nøyaktig resultater med et mer presist siktemiddel. Variasjonen mellom målingene med Tyholttårnet og Nidarosdomen som referansepunkt tyder på at et mer nøyaktig siktemiddel ville gitt bedre resultater.

\end{itemize}



 