% Abstract -> Sammendrag
%\abstracttitle{Sammendrag} % Spesifikt for elsarticle

\section*{Sammendrag}

Magnetisk deklinasjon er en misvisning som kommer av at jordas magnetfelt ikke er orientert langs jordas 
rotasjonakse, og inklinasjon er vinkel mellom horisontalplanet og magnetfeltet. For å finne inklinasjonen, kan det gjøres målinger av magnetfeltet og så analysere disse. Ved å måle magnetfeltet i forhold til referansepunkter, kan 
deklinasjonen bestemmes ut fra disse dersom geografisk nord også er kjent i forhold til referansepunktene. Resultatene ble en deklinasjon på $-6.14\pm{5.39}$\textdegree og inklinasjon på 
$71.20\pm{4.82}$\textdegree. De største feilkildene var mulig variasjon i koordinatsystem og unøyaktig sensor i telefonen 
og unøyaktig siktemiddel. Deklinasjonen og inklinasjonen er innenfor størrelsesområde til World Magnetic Model, men 
deklinasjonen har motsatt fortegn i forhold.


