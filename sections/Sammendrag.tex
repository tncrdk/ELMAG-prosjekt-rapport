% Abstract -> Sammendrag
%\abstracttitle{Sammendrag} % Spesifikt for elsarticle

\section*{Sammendrag}

Magnetisk deklinasjon er en misvisning som kommer av at jordas magnetfelt ikke er orientert langs jordas 
rotasjonakse. Den er definert som vinkelen mellom geografisk nord og retningen på magnetfeltet. Inklinasjon er vinkel mellom horisontalplanet og magnetfeltet. For å finne inklinasjonen kan 
det gjøres målinger av magnetfeltet og så analysere disse. Ved å måle magnetfeltet i forhold til 
referansepunkter kan 
deklinasjonen bestemmes ut fra disse dersom geografisk nord også er kjent i forhold til referansepunktene. 
Resultatene ble en deklinasjon på $6.14\pm{5.39}$\textdegree mot vest og inklinasjon på 
$71.20\pm{4.82}$\textdegree. De største feilkildene var mulig variasjon i koordinatsystem, feil med sensor i 
telefonen, og unøyaktig siktemiddel. Deklinasjonen og inklinasjonen er innenfor størrelsesområdet til World 
Magnetic Model, men 
deklinasjonen vi målte var mot vest i kontrast til World Magnetic Model, som oppgir den mot øst.


