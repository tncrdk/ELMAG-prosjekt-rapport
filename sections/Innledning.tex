\section{Innledning}

Historisk sett har variasjonen i den magnetiske deklinasjonen og inklinasjonen ført til komplikasjoner innen navigasjon. Magnetisk deklinasjon forteller noe om avviket mellom geografisk nord og magnetisk nord. Magnetisk inklinasjon forteller på den andre siden noe om avviket mellom horisontalplanet og den jordmagnetiske feltvektoren. Disse to fenomenene vil være avhengig av hvor på jorden man befinner seg.  

Det finnes flere måter å måle magnetisk deklinasjon på. I dag brukes en kombinasjon av flere forskjellige typer utstyr, men den mest fremtredende og nøyaktige måten er ved bruk av satelitter \cite{World_magnetic_model}.     

Den magnetiske inklinasjonen kan måles ved hjelp av et "inklinatorium". Dette er en kompassnål som opererer i det vertikale planet, i motsetning til et kompass som opererer i horisontalplanet \cite{inklinometer}. 

I denne rapporten er målet å finne den magnetiske deklinasjonen og inklinasjonen ved hjelp av sensorene i mobiltelefonen. Det tas utgangspunkt i posisjonen til der målingen holdt sted, samt ett referansepunkt for å finne geografisk nord. Ut ifra denne informasjonen kan man bruke sensorene i mobiltelefonen til å finne magnetisk nord. Deretter kan man regne ut magnetisk deklinasjon og inklinasjon. Rapporten består av teorien og metoden som ligger bak målingene og utregningene. Deretter vil resultatetene for målingene bli fremlagt. Til slutt vil en diskusjon av resultatene og feilkildene bli lagt fram. 