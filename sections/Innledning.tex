\section{Innledning}

I flere århundre har jordens magnetfelt blitt brukt som et nyttig verktøy for å navigere seg rundt i verden. Kompasset kan dateres tilbake til så tidlig som 700-tallet, da araberne brukte det om bord på skipene sine \cite{kompass}. På lik linje som i dag var magnetisk deklinasjon, avviket mellom geografisk nord og magnetisk nord, samt magnetisk inklinasjon, avviket mellom horisontalplanet og den jordmagnetiske feltvektoren, fenomener som man måtte ta hensyn til om man skulle navigere seg rundt korrekt. Det var lenge uenighet i hvorfor det ofte var et avvik mellom det kompassnålen viste og det sanne nord. Enkelte hevdet at det var feil med selve kompasset. Det var ikke før etter Kristoffer Columbus sin kjente sjøreise til Amerika i 1492 at man fant ut at avviket ikke lå hos instrumentmakerne men var noe som var utenfor menneskenes hender \cite{kompass}. 

Det finnes flere måter å måle magnetisk deklinasjon på. Fram til 1990-tallet ble instrumentet "Deklinatorium" brukt til å bestemme den magnetiske deklinasjonen. Dette fungerte ved å ha en kompassnål hengende mest mulig friksjonsfritt for å måle det magnetiske nord. For å bestemme det sanne nord måler man posisjonen til eksempelvis solen eller stjerner gjennom en astronomisk observasjon. Vinkelen mellom disse to målingene gir deklinasjonen \cite{deklinatorium}. I dag brukes en kombinasjon av flere forskjellige typer utstyr, men den mest fremtredende og mest nøyaktige måten er ved bruk av satelitter \cite{World_magnetic_model}.    

Den magnetiske inklinasjonen kan måles ved hjelp av et "inklinatorium". Dette er et instrument som inneholder en magnetnål som opererer i det vertikale planet, i motsetning til et kompass som opererer i horisontalplanet. Ved å stille inn omdreiningsaksen vinkelrett på den magnetiske meridianen, halvsirkelen fra pol til pol, kan man lese av på instrumentet den magnetiske inklinasjonen \cite{inklinometer}. 

I denne rapporten er målet å finne den magnetiske deklinasjonen og inklinasjonen ved hjelp av sensorene i mobiltelefonen. Det tas utgangspunkt i posisjonen til der målingen holdt sted, samt et referansepunkt for å finne geografisk nord. Ut ifra denne informasjonen kan man bruke sensorene i mobiltelefonen til å finne magnetisk nord. Deretter kan man regne ut magnetisk deklinasjon og inklinasjon. Rapporten består av teorien og metoden som ligger bak målingene og utregningene. Deretter vil resultatetene for målingene bli fremlagt. Til slutt vil en diskusjon av resultatene og feilkildene bli lagt fram. \newline 

FOKORTET VERSJON - FORTSATT ROM FOR FORBEDRINGER \newline

Historisk sett har magnetisk deklinasjon, avviket mellom geografisk nord og magnetisk nord, samt magnetisk inklinasjon, avviket mellom horisontalplanet og den jordmagnetiske feltvektoren gjort det litt vanskeligere å navigere seg rundt på land og vann.

Det finnes flere måter å måle magnetisk deklinasjon på. I dag brukes en kombinasjon av flere forskjellige typer utstyr, men den mest fremtredende og mest nøyaktige måten er ved bruk av satelitter \cite{World_magnetic_model}.

Den magnetiske inklinasjonen kan måles ved hjelp av et "inklinatorium". Dette er en kompassnål som opererer i det vertikale planet, i motsetning til et kompass som opererer i horisontalplanet \cite{inklinometer}.

I denne rapporten er målet å finne den magnetiske deklinasjonen og inklinasjonen ved hjelp av sensorene i mobiltelefonen. Det tas utgangspunkt i posisjonen til der målingen holdt sted, samt ett referansepunkt for å finne geografisk nord. Ut ifra denne informasjonen kan man bruke sensorene i mobiltelefonen til å finne magnetisk nord. Deretter kan man regne ut magnetisk deklinasjon og inklinasjon. Rapporten består av teorien og metoden som ligger bak målingene og utregningene. Deretter vil resultatetene for målingene bli fremlagt. Til slutt vil en diskusjon av resultatene og feilkildene bli lagt fram. 