\section{Konklusjon}
Her kommer konklusjonen.

\begin{itemize}
    \item Kort oppsummering av forsøket - hva man har gjort, metodene man har bruk og hensikten med forsøket. 
    \item oppsummering av slutningene man har kommet frem til.
    \item De viktigste punktene fra diskusjonsdelen - de største feilkildene. 
    \item Hva kunne man ha gjort annerledes?
\end{itemize}

Det ble gjort målinger for å finne den magnetiske deklinasjonen og inklinasjon. Målingene ble gjennomført med bruk av telefoner og Phyphox. Data ble så eksportert og analysert ved hjelp av Python. For å finne deklinasjonen ble det valgt 2 referansepunkter som et koordinatsystem. Ved å analysere dataene og så sammenligne de med referansepunktenes vinkel til geografisk nord, kunne deklinasjonen bestemmes. Inklinasjonen ble bestemt ved å beregne vinkelen mellom horisontalplanet og vertikalplanet til magnetfeltet. 

Den gjennomsnittlige magnetiske deklinasjonen ble målt til $-6.14 \pm 5.39$\textdegree. \ Den gjennomsnittlige magnetiske inklinasjonen ble målt til $71.20 \pm 4.82$\textdegree. Det største bidraget til usikkerheten kan ha vært siktemiddelet, da dette er utsatt for menneskelig feil. I tillegg er det mulig at Phyphox sitt referansesystem kan være feil, og dette vil trolig bidra til systematiske feil. Ved en senere gjentakelse av forsøket vil det være hensiktsmessig å gjøre flere målinger over flere dager.     
